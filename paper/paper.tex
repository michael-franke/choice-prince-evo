\documentclass[fleqn,reqno,11pt]{article}

%========================================
% Packages
%========================================

\usepackage[]{helpers/mypackages}
%\usepackage[natbib=true,style=authoryear-comp,backend=bibtex,doi=false,url=false]{biblatex}
\bibliography{helpers/MyRefLocal}
\usepackage{helpers/myenvironments}
\usepackage{helpers/mycommands}

\usepackage{todonotes}

%========================================
% Standard Layout
%========================================

\usepackage{a4wide}

% Itemize
\renewcommand{\labelitemi}{\large{$\mathbf{\cdot}$}}    % itemize symbols
\renewcommand{\labelitemii}{\large{$\mathbf{\cdot}$}}
\renewcommand{\labelitemiii}{\large{$\mathbf{\cdot}$}}
\renewcommand{\labelitemiv}{\large{$\mathbf{\cdot}$}}
% Description
\renewcommand{\descriptionlabel}[1]{\hspace\labelsep\textsc{#1}}

% Figure Captions
\usepackage{caption} % use corresponding myfiguresize!
\setlength{\captionmargin}{20pt}
\renewcommand{\captionfont}{\small}
\setlength{\belowcaptionskip}{7pt} % standard is 0pt

\newcommand{\myalert}[1]{\textcolor{red}{#1}}

\title{Rationality, Evolution \& the Environment}
\author{Paolo Galeazzi \& Michael Franke}
\date{current version: May 27th 2015}

\begin{document}
\maketitle


\section{Introduction / Motivation}

\begin{itemize}
\item Evolutionary game theory has become an established philosophical tool for probing into
  conceptual issues whose complexity requires mathematical modeling.
  \begin{itemize}
  \item helped structure debate about foraging, mating, evolution of cooperation,
    information-flow, deception etc.
  \end{itemize}
\item Most applications of EGT look at evolutionary selection of behavior in a single
  game. There are two crude simplifications here:
  \begin{enumerate}
  \item The stage game is usually considered fixed, closed and immutable.
    \begin{itemize}
    \item there are some notable exceptions where the game is open ended, e.g., by the
      invention of actions \citep{WordenLevin2007:Evolutionary-es,
        McKenzie-AlexanderSkymrs2012:Inventing-New-S}
    \end{itemize}
  \item What evolves are behavioral strategies for the stage game \citep[what][call the
    ``behavioral gamit'']{FawcettHamblin2013:Exposing-the-be} \myalert{(more references!
      Gigerenzer etc.)}.
  \end{enumerate}
\item It has been argued that we should rather study the evolutionary selection of \emph{choice
    mechanisms} that are adapted in a rich and variable environment
  \citep[e.g.][]{FawcettHamblin2013:Exposing-the-be,McNamara2013:Towards-a-Riche}.
\item However, studies on the adaptive value of choice mechanisms have focused, for the most
  part, of the evolutionary benefit of learning rules
  \citep[e.g.][]{ZollmanSmead2010:Plasticity-and-,SmeadZollman2013:The-Stability-o}. (\myalert{More
    references Harley 1981!!!, \dots}). Moreover, only little attention has been paid to
  modeling the direct (game-theoretic) meta-competition between alternative choice mechanisms
  \citep[see][for related criticism]{FawcettHamblin2013:Exposing-the-be}.
\item Here, we would like to make two contributions to the literature on evolution of choice
  mechanisms:
  \begin{enumerate}
  \item On the conceptual side, we argue that paying attention to the evolution of (subjective
    representations of) preferences matters. Since almost any reasonable learning or
    decision-making process will make use of some (subjective, agent-internal (possibly
    unconscious)) representation of the quality of choice options, the question of which such
    representations are ecologically valuable (i.e., lead to high fitness) is central. In fact,
    the evolution of (subjective representations of) preferences has recently been studied in
    theoretical economics as well \myalert{add references}. However, these studies also suffer
    from the ``behavioral gambit'' and the usual shortcomings of the closedness of the stage game. 
  \item On the methodological side, we therefore introduce a conservative extension of the
    standard EGT that allows us to reuse notions such as evolutionary stability / evolutionary
    dynamics and yet helps us study the evolutionary competition between choice mechanisms in a
    statistically variable environment. Concretely, we look at the evolutionary competition of
    choice mechanisms in a ``meta-game'' that consists of the average expected payoffs of
    choice mechanisms when playing arbitrary games (from a given class). In this sense, a
    meta-game captures (the modeller's) assumptions about the relevant statistic of the
    environment in which evolutionary forces operate.
  \end{enumerate}
\item To drive home our conceptual point that preference representations can matter, we try to
  isolate, as good as we can, the evolution of preferences from considerations of learning or
  other factors feeding a decision mechanism (such as the acquisition of ecologically useful
  representations (think: beliefs)). (Obviously, eventually, we would like to study the
  evolution of all of these relevant components in one swoop.)
\item To demonstrate how this approach can be theoretically rewarding, we show that, under
  reasonable additional assumptions, evolutionary selection can support non-veridical
  representations of payoffs in terms of regret.
  \begin{itemize}
  \item We argue that this ``meta-evolutionary'' approach could eventually help explain
    attested deviations of human decision making from the classical ideal: choice behavior
    evolved to be an ``ecologically rational'', near-optimal adaptation to the environment
    \citep{Anderson1990:The-Adaptive-Ch,Anderson1991:Is-human-cognit,GigerenzerGoldstein1996:Reasoning-the-F,ChaterOaksford2000:The-Rational-An}. This
    is inline with recent aspirations to provide an evolutionary rationale for decision making
    processes
    \citep[e.g.][]{HammersteinStevens2012:Six-Reasons-for,FawcettHamblin2013:Exposing-the-be}.
  \item Non-veridical regret-based representations of payoffs are intuitively appealing (we do,
    at least on occasion, care about counterfactual or hypothetical losses), but could also
    help explain, for instance, that decision-makers may violate the independence axiom: human
    and non-human decision makers can favor $A$ when presented with a choice
    between $A$ and $B$, but prefer $B$ when presented with $A$, $B$ and $C$ \myalert{(for
      animals cite from Fawcett page 3)} \myalert{(add
      citation for human performance)}
  \item We need to be careful here: we do not want to stress that this approach here (the
    particular implemenation of a regret-based choice mechanism) explains these empirical
    findings (better than any other line of explanation). Our point is more general and
    methodological. Evolution of preferences is something that the literature on evolution of
    choice mechanisms should care about. We also presented a method that can be useful and an
    example of that method that should (succinctly) demonstrate our aforementioned conceptual
    point that evolution of preferences is conceptually important and can give non-trivial,
    unexpected and philosophically useful results.
  \end{itemize}
\item contributes to: (i) methodological reflection on evolutionary game theory, (ii) ecological
  rationality of choice behavior, (iii) evolution of preferences literature
\end{itemize}

\newpage 


\section{Faces of rationality}


VERSION 1

A central pivot in economics and decision theory is the definition of rationality as maximization of subjective preferences. The subjectivity of preferences plays a very important role. From an agent's standpoint, the amount of, say, money achieved does not necessarily coincide with the amount of subjective utility attained. The first to notice this difference between objective utility (goods, money, etc.) and subjective utility (preferences) was D. Bernoulli in the famous St. Petersburg paradox. Moreover, Bernoulli introduced the idea that subjective utility can be represented by an increasing concave function of objective utility. This can be considered the birth of modern decision theory.
Another important step was developed by Kahnemann and Tversky who introduced prospect theory in their famous paper from 1979. By means of experimental results they noticed that the decision maker apparently acts as a risk averse agent with respect to gains and as a risk loving agent with respect to losses. Consequently, they conjectured that the subjective utility is better representable by an S-shaped utility function, concave in the positive domain of gains and convex in the negative domain of losses.
More recently, other possible definitions of the subjective utility function have been proposed and justified from a game theoretical point of view (e.g. Fehr and Schmidt, Charness and Rabin, etc.). The central goal of these studies is to explain/describe the behavior in social dilemmas (e.g., Prisoner's dilemma, public goods game, etc.). By means of different subjective utility functions, it is then possible to describe/match the behavior observed in experiments and to explain it as rational (in the sense of maximizing the subjective utility). \\
With this approach it seems that every behavior, no matter how far from the maximization of objective utility, can be deemed as rational if we can only find a subjective utility that justifies that behavior as maximizing the subjective utility. Quoting Hegel, (almost) everything real is rational. The problem with this approach is that it seems that everything that is observed can be considered rational, given the adequate subjective utility. The notion of rationality almost collapses onto the observed behavior.

On the other side, other economists (Alchian, Friedman) seem to link the notion of rationality more tightly to the maximization of objective utility. In their classic works they argue that profit maximization is a reasonable assumption for characterizing outcomes in a competitive market because only firms behaving in a manner that is consistent with profit maximization will survive in the long run. If the previous accounts of rationality looked more descriptive, this definitely tastes more normative. In this sense, the normative notion of rationality is tightly linked to (and almost coincide with) the maximization of objective utility, which is supposed to drive the evolutionary process. It follows then that being normative in this sense seems to entail an identification of rationality with maximization of objective utility.

It seems that we are stuck between two extrema. On the one hand, if we want to have a descriptive notion of rationality, then it seems we can (almost) always define an appropriate subjective utility whose maximization matches the observed behavior. On the other hand, if we want to have a normative approach to rationality, then it seems we have to appeal to evolutionary arguments implying the understanding of rationality as maximization of objective utility.

The meta-game perspective that we introduce here can shed more light on this duality, and opens some space in between the two extrema. By keeping the evolutionary approach we want to retain a normative perspective on rationality, but our results show that (under certain conditions) evolution does not necessarily lead to identifying rationality with maximization of objective utility. More precisely, from the meta-game perspective many possible subjective transformations of the objective utility are eliminated by evolution, but we find some evolutionarily stable subjective utility maximizers whose utility does not coincide with the objective evolutionary utility. This result leaves some room in between the two extrema: we can understand rationality in a normative way, given by evolution, without having to discard all the possible subjective utility representations as irrational.

In particular, in the following we mainly deal with two types of agents: players whose utility function coincides with the objective utility, and players whose subjective utility is defined in terms of regret. The justification for this choice is both historical and behavioral. Among the possible competitors of utility maximization, regret has always been in a prominent position, both from a descriptive and normative viewpoint. Indeed, Bell82 writes 
\begin{quote}
By explicitly incorporating regret, expected utility theory not only becomes a better descriptive predictor but also may become a more convincing guide for prescribing behavior to decision makers.
\end{quote}

\citet{LoomesSugden1982:Regret-Theory:-} show that a regret-based theory of choice can match and explain the main experimental results of prospect theory, and claim that
\begin{quote}

[...] we shall challenge the idea that the conventional axioms constitute the only acceptable basis for rational choice under uncertainty. We shall argue that it is no less rational to act in accordance with regret theory, and that conventional expected utility theory therefore represents an unnecessarily restrictive notion of rationality. \cite{loosug82}

\end{quote}

...

END VERSION 1





VERSION 2

The importance of aiming at a richer evolutionary game theory has been recently emphasized by many works in theoretical biology and behavioral ecology [Fawcett\&al., McNamara, etc.]. The biological argument to embrace this goal is related to the criticisms towards the so-called behavioral gambit [Fawcett\&al.]. The behavioral gambit denotes the standard approach in evolutionary game theory that puts the focus of attention on the expressed behavior, and neglects the underlying mechanisms that generate that behavior. From this point of view, the benefit and the reliability coming from the use of evolutionary game theory for biological purposes is also under attack. There are good reasons to claim that focusing on the general psychological mechanisms instead of on behavior itself as the phenotype under evolutionary selection can be more faithful and insightful for natural biology and ecology [Fawcett\&al.].
Standard evolutionary game theory normally models the agents as playing a single fixed game, and as genetically predetermined to play fixed strategies, that consequently are the only target of natural selection in those frameworks. Standard models of evolutionary game theory cannot take charge of this shift of perspective until they will keep modelling a single game at a time, and until they will keep focusing on actions considered genetically encoded and fixed. ''Instead we should expect animals to have evolved a set of psychological mechanisms which enable them to perform well on average across a range of different circumstances''[Fawcett\&al.].

Attempts to overcome the behavioral gambit have recently been developed in economic literature too. In the last twenty years economists have been using evolutionary game-theoretical models to study evolution of preferences [cite], instead of evolution of behavior only. This approach is called indirect evolutionary approach, because players' behavior and actions stem from their subjective preferences, and it is aimed at investigating what subjective preferences are evolutionarily better off and robust with respect to natural selection. This shift from a direct evolutionary approach about observed actions and behavior to an indirect evolutionary approach takes only partly into account the behavioral gambit criticized by biologists and ecologists. The reason is that the models of evolution of preferences usually consider only one game at a time. This means that there is a single type of interaction that is supposed to drive the whole evolution of preferences: preferences are selected on the basis of one possible interaction only. Clearly, this approach does not take into consideration the variety of possible circumstances that animals are supposed to face during their lifetime. 

The meta-game model we propose pays heed to both components of the behavioral gambit. We retain the indirect evolutionary approach, but our agents in the population play over a class of possible games $\mathcal{G}$. In this way we explicitly model a range of different circumstances, as wished by theoretical biologists.  Consequently, each agent in the population does not represent either a simple behavior or a subjective preference, but a more general player type that associates a subjective preference for each game in the class $\mathcal{G}$. The player type is supposed to encode the general psychological mechanisms that generate the behavior for any possible interaction in $\mathcal{G}$. These general psychological mechanisms then become the target of evolution, and they are selected based on the average fitness over a class of different interactions. 

In particular, in the following we mainly deal with two types of agents: players whose subjective preference coincides with the evolutionary fitness, and players whose subjective preference is defined in terms of regret. It turns out that psychological mechanisms and subjective conceptualizations based on regret can outperform more veridical conceptualizations coinciding with the evolutionary fitness.
This is just an example of the kind of insights and results that we can gain by adopting a meta-game perspective and by shifting the focus of our attention to psychological mechanisms and subjective conceptualizations as the phenotype under selection.


-maybe to add: difference with learning. Fawcett\&al. talk more about differences in learning, whereas we model differences in subjective conceptualizations.


END VERSION 2


\begin{itemize}
  \item \todo[inline]{choose uniform consensual terminology for maximization of EU and regret minimization}
\item classical notion of rationality: maximization of expected utilities
  \begin{itemize}
  \item human choice behavior seems to deviate from classical rational choice in many ways
    \citep[most
    notably][]{TverskyKahnemann1974:Judgement-under,KahnemannTversky1979:Prospect-Theory}
  \end{itemize}
\item historical competitor: regret minimization
  \begin{itemize}
  \item introduced by Savage \& Niehans independently
  \item worked out (in parameterized form) by \citet{LoomesSugden1982:Regret-Theory:-} and
    argued to outperform prospect theory
  \item recently rediscovered by game theorists \citep{HalpernPass2012:Iterated-Regret} as a
    possible partial explanation of some puzzles (e.g., Travellers Dilemma)
  \end{itemize}
\item motivate why we are interested in regret minimization; e.g.:
  \begin{itemize}
  \item simple historical alternative; certain prima facie plausibility
  \item has both components of interest:
    \begin{itemize}
    \item epistemic component: probabilistic vs.~non-probabilistic uncertainty  (interesting
      because of the classical Ellsberg paradox)
    \item payoff component: veridical vs.~non-veridical preference representations (interesting
      because of a recent interest in the ``evolution of preference'')
    \item gives us intriguing results that nicely motivate the general approach of meta-games
    \end{itemize}
  \end{itemize}
\end{itemize}

\section{Evolution \& the environment}

here we spell out our model

\subsection{Choice principles}

\begin{itemize}
\item study evolutionary competition between \emph{choice principles} 
\item for now a choice principle is a pair (epistemic type, preference type)
  \begin{itemize}
  \item two epistemic types:
    \begin{enumerate}
    \item probabilistic beliefs (assumed to be flat)
    \item non-probabilistic beliefs
    \end{enumerate}
  \item two preferences types (references to ``evolution of preference'' literature):
    \begin{enumerate}
    \item veridical payoff representations
    \item regret-based payoffs
    \end{enumerate}
  \item many other types are conceivable; more in discussion section
    \begin{itemize}
    \item we want to keep it simple
    \item assume that agents are relatively simple as well
    \item no learning from past encounters \& no reasoning about the opponent: agents simply do
      not know that there is another player (no strategic thinking possible; in line with idea
      of evolutionary game theory where agents are ``stupid'' and ``myopic'') \\
      \textcolor{blue}{this is then the added assumption that may be crude and generally
        implausible but allows us to deflect the economists' critique}
    \end{itemize}
  \item choice principles yield action choices for each game
    \begin{itemize}
    \item show how
    \end{itemize}
  \item RESULT: types (prob., veridical) and (prob., regret) are choice-equivalent (also
    mentioned by \citet{HalpernPass2012:Iterated-Regret})
  \item RESULT: in $2 \times 2$ cases, (prob., veridical), (prob., regret) and (non-prob.,
    regret) are choice-equivalent (but not for $n \times n$ in general)
    \begin{itemize}
    \item still we treat them as separate types because they behave differently under
      component-wise mutation (see below)
    \end{itemize}
  \end{itemize}
\end{itemize}

\subsection{The environment: meta-games}

\begin{itemize}
\item imagine huge, virtually infinite population of agents, each carrying a choice principle
\item agents play a randomly sampled game against a randomly sampled opponent
\item fitness is based on the average payoff of choice principle A against choice principle B,
  taken over arbitrary games $g$ and weighted their occurrence probability $P(g)$:
  $\int_g P(g) \times U_g(\text{CP-A}, \text{CP-B}) \text{d}g$
\item the occurrence probability of games $g$ is the ``environment'' and, possibly, impossible to
  derive theoretically, or to measure empirically
\item here, use a relatively neutral approach instead that is practicable:
  \begin{itemize}
  \item games are individuated by their payoff matrices (no framing effects or similar)
  \item each entry in a payoff matrix is an i.~i.~d.~random variable
    \begin{itemize}
    \item approach still incorporates a bias (e.g., against strategic form representations of
      sequential games)
    \end{itemize}
  \end{itemize}
\item example: average payoffs from numerical simulation of ($2 \times 2$ symmetric games)
\item example: average payoffs from numerical simulation of ($n \times n$ symmetric games, with
  $n$ uniformly random (up to some upper-bound))
\end{itemize}

\subsection{Results}

\textcolor{gray}{keep this clean and concise, formal details (proofs) in appendix}

\subsubsection{$2 \times 2$ symmetric games}

\textcolor{gray}{we isolate this case because it is most basic, and we can offer analytical
  proofs as well}

\begin{itemize}
\item given previous results, there are really only two kinds of types to look at:
  \begin{enumerate}
  \item (prob., veridical), (prob., regret) and (non-prob., regret) \\ 
    \textcolor{gray}{[all choice-equivalent in this case]}
  \item (non-prob., veridical)
  \end{enumerate}
\item RESULT: the former class ``dominates'' the latter (our main result, Proposition 1 from
  TARK paper)
\item this means that evolutionary dynamics would weed out (non-prob., regret); every
  population composition between the remaining three is neutrally stable; all interior
  trajectories (of the replicator dynamic) converge to these
\item further results (from TARK):
  \begin{itemize}
  \item component-wise mutation: abundance of regret types
    \begin{itemize}
    \item different evolutionary paths despite the same behavior (contra the ``behavioral
      gambit'' \citep{FawcettHamblin2013:Exposing-the-be})
    \end{itemize}
  \item exogenously given correlated epistemic types: regret types dominate
  \item exogenously given uncorrelated epistemic types: regret types dominate
  \end{itemize}
\todo[inline]{work out these additional results more carefully; present graphs if possible}
\end{itemize}

\subsubsection{$n \times n$ symmetric games}

\todo[inline]{work out these additional results more carefully; present graphs if possible}

\section{Discussion}

\begin{itemize}
\item what we achieved
\item what could be done next
\end{itemize}




\printbibliography[heading=bibintoc]


\newpage

\section*{Snippets}

\begin{itemize}
\item We present an approach to studying evolutionary selection of \emph{choice principles},
  i.e., ways of choosing consistently across several games. In other words, we would like to
  look at the problem of ``rational choice'' from an evolutionary point of view: what choice
  principles would be promoted, which demoted in evolutionary competition amongst each other?

\item This depends on the environment; it depends on what choice situations occur how
  frequently. Maybe some choice principle is better in some situations, while another one is
  better in others. 


\item We argue that this ``meta-evolutionary'' approach could eventually help explain attested
  deviations of human decision making from the classical ideal: choice behavior evolved to be
  an ``ecologically rational'', near-optimal adaptation to the environment
  \citep{Anderson1990:The-Adaptive-Ch,Anderson1991:Is-human-cognit,GigerenzerGoldstein1996:Reasoning-the-F,ChaterOaksford2000:The-Rational-An}. This
  is inline with recent aspirations to provide an evolutionary rationale for decision making
  processes
  \citep[e.g.][]{HammersteinStevens2012:Six-Reasons-for,FawcettHamblin2013:Exposing-the-be}. Here,
  we seek to contribute to the formal machinery, an extension of evolutionary game theory, that
  allows to capture statistical properties of the environment
  \citep[cf.][]{McNamara2013:Towards-a-Riche}.
\end{itemize}

\end{document}



