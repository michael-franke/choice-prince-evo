\documentclass[fleqn,reqno,11pt]{article}

%========================================
% Packages
%========================================

\usepackage[]{helpers/mypackages}
%\usepackage[natbib=true,style=authoryear-comp,backend=bibtex,doi=false,url=false]{biblatex}
\bibliography{helpers/MyRefLocal}
\usepackage{helpers/myenvironments}
\usepackage{helpers/mycommands}

\usepackage{todonotes}

%========================================
% Standard Layout
%========================================

\usepackage{a4wide}

% Itemize
\renewcommand{\labelitemi}{\large{$\mathbf{\cdot}$}}    % itemize symbols
\renewcommand{\labelitemii}{\large{$\mathbf{\cdot}$}}
\renewcommand{\labelitemiii}{\large{$\mathbf{\cdot}$}}
\renewcommand{\labelitemiv}{\large{$\mathbf{\cdot}$}}
% Description
\renewcommand{\descriptionlabel}[1]{\hspace\labelsep\textsc{#1}}

% Figure Captions
\usepackage{caption} % use corresponding myfiguresize!
\setlength{\captionmargin}{20pt}
\renewcommand{\captionfont}{\small}
\setlength{\belowcaptionskip}{7pt} % standard is 0pt



\title{Rationality, Evolution \& the Environment}
\author{Paolo Galeazzi \& Michael Franke}
\date{current version: May 18th 2015}

\begin{document}
\maketitle


\section*{Outline of main argument}


\begin{itemize}
\item Evolutionary game theory has become an established philosophical tool for probing into
  conceptual issues whose complexity requires mathematical modeling. Most applications of EGT
  look at evolutionary selection of behavior in a single game. This game is then considered
  fixed, closed and immutable.
  \begin{itemize}
  \item there are some notable exceptions where the game is open ended, e.g., by the invention
    of actions \citep{WordenLevin2007:Evolutionary-es, McKenzie-AlexanderSkymrs2012:Inventing-New-S}
  \end{itemize}

\item We present an approach to studying evolutionary selection of \emph{choice principles},
  i.e., ways of choosing consistently across several games. In other words, we would like to
  look at the problem of ``rational choice'' from an evolutionary point of view: what choice
  principles would be promoted, which demoted in evolutionary competition amongst each other?

\item This depends on the environment; it depends on what choice situations occur how
  frequently. Maybe some choice principle is better in some situations, while another one is
  better in others. To capture the relevant statistics of the environment, we therefore look at
  the evolutionary competition of choice principles in a ``meta-game'' that consists of the
  average expected payoffs of choice principles when playing arbitrary games (from a given
  class). In this sense, a meta-game captures (the modeller's) assumptions about the relevant
  statistic of the environment in which evolutionary forces operate.

\item To demonstrate how this approach can be theoretically rewarding, we show that, under
  reasonable additional assumptions, maximization of expected utility is not the only choice
  principle supported by evolutionary selection. In particular, we show that evolutionary
  selection can support non-veridical representations of payoffs in terms of regret. 

\item We argue that this ``meta-evolutionary'' approach could eventually help explain attested
  deviations of human decision making from the classical ideal: choice behavior evolved to be
  an ``ecologically rational'', near-optimal adaptation to the environment
  \citep{Anderson1990:The-Adaptive-Ch,Anderson1991:Is-human-cognit,GigerenzerGoldstein1996:Reasoning-the-F,ChaterOaksford2000:The-Rational-An}. This
  is inline with recent aspirations to provide an evolutionary rationale for decision making
  processes
  \citep[e.g.][]{HammersteinStevens2012:Six-Reasons-for,FawcettHamblin2013:Exposing-the-be}. Here,
  we seek to contribute to the formal machinery, an extension of evolutionary game theory, that
  allows to capture statistical properties of the environment
  \citep[cf.][]{McNamara2013:Towards-a-Riche}.

\end{itemize}


\newpage 

\section{Introduction / Motivation}

\begin{itemize}
\item evolutionary game theory about one game usually
\item here class of games
\item therefore partial solution to the ``closed game'' problem
\item still not infinite, open ended strategy space
\item focus on one example for this new method: competition between:
  \begin{itemize}
  \item probabilistic vs.~non-probabilistic uncertainty
  \item veridical vs.~regret-based preferences
  \end{itemize}
\item contributes to: (i) methodological reflection on evolutionary game theory, (ii) ecological
  rationality of choice behavior, (iii) evolution of preferences literature
\end{itemize}

\section{Faces of rationality}

\begin{itemize}
  \item \todo[inline]{choose uniform consensual terminology for maximization of EU and regret minimization}
\item classical notion of rationality: maximization of expected utilities
  \begin{itemize}
  \item human choice behavior seems to deviate from classical rational choice in many ways
    \citep[most
    notably][]{TverskyKahnemann1974:Judgement-under,KahnemannTversky1979:Prospect-Theory}
  \end{itemize}
\item historical competitor: regret minimization
  \begin{itemize}
  \item introduced by Savage \& Niehans independently
  \item worked out (in parameterized form) by \citet{LoomesSugden1982:Regret-Theory:-} and
    argued to outperform prospect theory
  \item recently rediscovered by game theorists \citep{HalpernPass2012:Iterated-Regret} as a
    possible partial explanation of some puzzles (e.g., Travellers Dilemma)
  \end{itemize}
\item motivate why we are interested in regret minimization; e.g.:
  \begin{itemize}
  \item simple historical alternative; certain prima facie plausibility
  \item has both components of interest:
    \begin{itemize}
    \item epistemic component: probabilistic vs.~non-probabilistic uncertainty  (interesting
      because of the classical Ellsberg paradox)
    \item payoff component: veridical vs.~non-veridical preference representations (interesting
      because of a recent interest in the ``evolution of preference'')
    \item gives us intriguing results that nicely motivate the general approach of meta-games
    \end{itemize}
  \end{itemize}
\end{itemize}

\section{Evolution \& the environment}

here we spell out our model

\subsection{Choice principles}

\begin{itemize}
\item study evolutionary competition between \emph{choice principles} 
\item for now a choice principle is a pair (epistemic type, preference type)
  \begin{itemize}
  \item two epistemic types:
    \begin{enumerate}
    \item probabilistic beliefs (assumed to be flat)
    \item non-probabilistic beliefs
    \end{enumerate}
  \item two preferences types (references to ``evolution of preference'' literature):
    \begin{enumerate}
    \item veridical payoff representations
    \item regret-based payoffs
    \end{enumerate}
  \item many other types are conceivable; more in discussion section
    \begin{itemize}
    \item we want to keep it simple
    \item assume that agents are relatively simple as well
    \item no learning from past encounters \& no reasoning about the opponent: agents simply do
      not know that there is another player (no strategic thinking possible; in line with idea
      of evolutionary game theory where agents are ``stupid'' and ``myopic'') \\
      \textcolor{blue}{this is then the added assumption that may be crude and generally
        implausible but allows us to deflect the economists' critique}
    \end{itemize}
  \item choice principles yield action choices for each game
    \begin{itemize}
    \item show how
    \end{itemize}
  \item RESULT: types (prob., veridical) and (prob., regret) are choice-equivalent (also
    mentioned by \citet{HalpernPass2012:Iterated-Regret})
  \item RESULT: in $2 \times 2$ cases, (prob., veridical), (prob., regret) and (non-prob.,
    regret) are choice-equivalent (but not for $n \times n$ in general)
    \begin{itemize}
    \item still we treat them as separate types because they behave differently under
      component-wise mutation (see below)
    \end{itemize}
  \end{itemize}
\end{itemize}

\subsection{The environment: meta-games}

\begin{itemize}
\item imagine huge, virtually infinite population of agents, each carrying a choice principle
\item agents play a randomly sampled game against a randomly sampled opponent
\item fitness is based on the average payoff of choice principle A against choice principle B,
  taken over arbitrary games $g$ and weighted their occurrence probability $P(g)$:
  $\int_g P(g) \times U_g(\text{CP-A}, \text{CP-B}) \text{d}g$
\item the occurrence probability of games $g$ is the ``environment'' and, possibly, impossible to
  derive theoretically, or to measure empirically
\item here, use a relatively neutral approach instead that is practicable:
  \begin{itemize}
  \item games are individuated by their payoff matrices (no framing effects or similar)
  \item each entry in a payoff matrix is an i.~i.~d.~random variable
    \begin{itemize}
    \item approach still incorporates a bias (e.g., against strategic form representations of
      sequential games)
    \end{itemize}
  \end{itemize}
\item example: average payoffs from numerical simulation of ($2 \times 2$ symmetric games)
\item example: average payoffs from numerical simulation of ($n \times n$ symmetric games, with
  $n$ uniformly random (up to some upper-bound))
\end{itemize}

\subsection{Results}

\textcolor{gray}{keep this clean and concise, formal details (proofs) in appendix}

\subsubsection{$2 \times 2$ symmetric games}

\textcolor{gray}{we isolate this case because it is most basic, and we can offer analytical
  proofs as well}

\begin{itemize}
\item given previous results, there are really only two kinds of types to look at:
  \begin{enumerate}
  \item (prob., veridical), (prob., regret) and (non-prob., regret) \\ 
    \textcolor{gray}{[all choice-equivalent in this case]}
  \item (non-prob., veridical)
  \end{enumerate}
\item RESULT: the former class ``dominates'' the latter (our main result, Proposition 1 from
  TARK paper)
\item this means that evolutionary dynamics would weed out (non-prob., regret); every
  population composition between the remaining three is neutrally stable; all interior
  trajectories (of the replicator dynamic) converge to these
\item further results (from TARK):
  \begin{itemize}
  \item component-wise mutation: abundance of regret types
    \begin{itemize}
    \item different evolutionary paths despite the same behavior (contra the ``behavioral
      gambit'' \citep{FawcettHamblin2013:Exposing-the-be})
    \end{itemize}
  \item exogenously given correlated epistemic types: regret types dominate
  \item exogenously given uncorrelated epistemic types: regret types dominate
  \end{itemize}
\todo[inline]{work out these additional results more carefully; present graphs if possible}
\end{itemize}

\subsubsection{$n \times n$ symmetric games}

\todo[inline]{work out these additional results more carefully; present graphs if possible}

\section{Discussion}

\begin{itemize}
\item what we achieved
\item what could be done next
\end{itemize}




\printbibliography[heading=bibintoc]

\end{document}
