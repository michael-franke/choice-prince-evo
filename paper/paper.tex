\documentclass[fleqn,reqno,11pt]{article}

%========================================
% Packages
%========================================

\usepackage[]{helpers/mypackages}
%\usepackage[natbib=true,style=authoryear-comp,backend=bibtex,doi=false,url=false]{biblatex}
\bibliography{helpers/MyRefLocal}
\usepackage{helpers/myenvironments}
\usepackage{helpers/mycommands}

\usepackage{todonotes}

%========================================
% Standard Layout
%========================================

\usepackage{a4wide}

% Itemize
\renewcommand{\labelitemi}{\large{$\mathbf{\cdot}$}}    % itemize symbols
\renewcommand{\labelitemii}{\large{$\mathbf{\cdot}$}}
\renewcommand{\labelitemiii}{\large{$\mathbf{\cdot}$}}
\renewcommand{\labelitemiv}{\large{$\mathbf{\cdot}$}}
% Description
\renewcommand{\descriptionlabel}[1]{\hspace\labelsep\textsc{#1}}

% Figure Captions
\usepackage{caption} % use corresponding myfiguresize!
\setlength{\captionmargin}{20pt}
\renewcommand{\captionfont}{\small}
\setlength{\belowcaptionskip}{7pt} % standard is 0pt

\newcommand{\myalert}[1]{\textcolor{red}{#1}}

\title{Rationality, Evolution \& the Environment}
\author{Paolo Galeazzi \& Michael Franke}
\date{current version: May 18th 2015}

\begin{document}
\maketitle


\section*{Outline of main argument}


\begin{itemize}
\item Evolutionary game theory has become an established philosophical tool for probing into
  conceptual issues whose complexity requires mathematical modeling.
  \begin{itemize}
  \item helped structure debate about foraging, mating, evolution of cooperation,
    information-flow, deception etc.
  \end{itemize}
\item Most applications of EGT look at evolutionary selection of behavior in a single
  game. There are two crude simplifications here:
  \begin{enumerate}
  \item The stage game is usually considered fixed, closed and immutable.
    \begin{itemize}
    \item there are some notable exceptions where the game is open ended, e.g., by the
      invention of actions \citep{WordenLevin2007:Evolutionary-es,
        McKenzie-AlexanderSkymrs2012:Inventing-New-S}
    \end{itemize}
  \item What evolves are behavioral strategies for the stage game \citep[what][call the
    ``behavioral gamit'']{FawcettHamblin2013:Exposing-the-be} \myalert{(more references!
      Gigerenzer etc.)}.
  \end{enumerate}
\item It has been argued that we should rather study the evolutionary selection of \emph{choice
    mechanisms} that are adapted in a rich and variable environment
  \citep[e.g.][]{FawcettHamblin2013:Exposing-the-be,McNamara2013:Towards-a-Riche}.
\item However, studies on the adaptive value of choice mechanisms have focused, for the most
  part, of the evolutionary benefit of learning rules
  \citep[e.g.][]{ZollmanSmead2010:Plasticity-and-,SmeadZollman2013:The-Stability-o}. (\myalert{More
    references Harley 1981!!!, \dots}). Moreover, only little attention has been paid to
  modeling the direct (game-theoretic) meta-competition between alternative choice mechanisms
  \citep[see][for related criticism]{FawcettHamblin2013:Exposing-the-be}.
\item Here, we would like to make two contributions to the literature on evolution of choice
  mechanisms:
  \begin{enumerate}
  \item On the conceptual side, we argue that paying attention to the evolution of (subjective
    representations of) preferences matters. Since almost any reasonable learning or
    decision-making process will make use of some (subjective, agent-internal (possibly
    unconscious)) representation of the quality of choice options, the question of which such
    representations are ecologically valuable (i.e., lead to high fitness) is central. In fact,
    the evolution of (subjective representations of) preferences has recently been studied in
    theoretical economics as well \myalert{add references}. However, these studies also suffer
    from the ``behavioral gambit'' and the usual shortcomings of the closedness of the stage game. 
  \item On the methodological side, we therefore introduce a conservative extension of the
    standard EGT that allows us to reuse notions such as evolutionary stability / evolutionary
    dynamics and yet helps us study the evolutionary competition between choice mechanisms in a
    statistically variable environment. Concretely, we look at the evolutionary competition of
    choice mechanisms in a ``meta-game'' that consists of the average expected payoffs of
    choice mechanisms when playing arbitrary games (from a given class). In this sense, a
    meta-game captures (the modeller's) assumptions about the relevant statistic of the
    environment in which evolutionary forces operate.
  \end{enumerate}
\item To drive home our conceptual point that preference representations can matter, we try to
  isolate, as good as we can, the evolution of preferences from considerations of learning or
  other factors feeding a decision mechanism (such as the acquisition of ecologically useful
  representations (think: beliefs)). (Obviously, eventually, we would like to study the
  evolution of all of these relevant components in one swoop.)
\item To demonstrate how this approach can be theoretically rewarding, we show that, under
  reasonable additional assumptions, evolutionary selection can support non-veridical
  representations of payoffs in terms of regret.
  \begin{itemize}
  \item We argue that this ``meta-evolutionary'' approach could eventually help explain
    attested deviations of human decision making from the classical ideal: choice behavior
    evolved to be an ``ecologically rational'', near-optimal adaptation to the environment
    \citep{Anderson1990:The-Adaptive-Ch,Anderson1991:Is-human-cognit,GigerenzerGoldstein1996:Reasoning-the-F,ChaterOaksford2000:The-Rational-An}. This
    is inline with recent aspirations to provide an evolutionary rationale for decision making
    processes
    \citep[e.g.][]{HammersteinStevens2012:Six-Reasons-for,FawcettHamblin2013:Exposing-the-be}.
  \item Non-veridical regret-based representations of payoffs are intuitively appealing (we do,
    at least on occasion, care about counterfactual or hypothetical losses), but could also
    help explain, for instance, that decision-makers may violate the independence axiom: human
    and non-human decision makers can favor $A$ when presented with a choice
    between $A$ and $B$, but prefer $B$ when presented with $A$, $B$ and $C$ \myalert{(for
      animals cite from Fawcett page 3)} \myalert{(add
      citation for human performance)}
  \item We need to be careful here: we do not want to stress that this approach here (the
    particular implemenation of a regret-based choice mechanism) explains these empirical
    findings (better than any other line of explanation). Our point is more general and
    methodological. Evolution of preferences is something that the literature on evolution of
    choice mechanisms should care about. We also presented a method that can be useful and an
    example of that method that should (succinctly) demonstrate our aforementioned conceptual
    point that evolution of preferences is conceptually important and can give non-trivial,
    unexpected and philosophically useful results.
  \end{itemize}







\end{itemize}


\newpage 

\section{Introduction / Motivation}

\begin{itemize}
\item evolutionary game theory about one game usually
\item here class of games
\item therefore partial solution to the ``closed game'' problem
\item still not infinite, open ended strategy space
\item focus on one example for this new method: competition between:
  \begin{itemize}
  \item probabilistic vs.~non-probabilistic uncertainty
  \item veridical vs.~regret-based preferences
  \end{itemize}
\item contributes to: (i) methodological reflection on evolutionary game theory, (ii) ecological
  rationality of choice behavior, (iii) evolution of preferences literature
\end{itemize}

\section{Faces of rationality}

\begin{itemize}
  \item \todo[inline]{choose uniform consensual terminology for maximization of EU and regret minimization}
\item classical notion of rationality: maximization of expected utilities
  \begin{itemize}
  \item human choice behavior seems to deviate from classical rational choice in many ways
    \citep[most
    notably][]{TverskyKahnemann1974:Judgement-under,KahnemannTversky1979:Prospect-Theory}
  \end{itemize}
\item historical competitor: regret minimization
  \begin{itemize}
  \item introduced by Savage \& Niehans independently
  \item worked out (in parameterized form) by \citet{LoomesSugden1982:Regret-Theory:-} and
    argued to outperform prospect theory
  \item recently rediscovered by game theorists \citep{HalpernPass2012:Iterated-Regret} as a
    possible partial explanation of some puzzles (e.g., Travellers Dilemma)
  \end{itemize}
\item motivate why we are interested in regret minimization; e.g.:
  \begin{itemize}
  \item simple historical alternative; certain prima facie plausibility
  \item has both components of interest:
    \begin{itemize}
    \item epistemic component: probabilistic vs.~non-probabilistic uncertainty  (interesting
      because of the classical Ellsberg paradox)
    \item payoff component: veridical vs.~non-veridical preference representations (interesting
      because of a recent interest in the ``evolution of preference'')
    \item gives us intriguing results that nicely motivate the general approach of meta-games
    \end{itemize}
  \end{itemize}
\end{itemize}

\section{Evolution \& the environment}

here we spell out our model

\subsection{Choice principles}

\begin{itemize}
\item study evolutionary competition between \emph{choice principles} 
\item for now a choice principle is a pair (epistemic type, preference type)
  \begin{itemize}
  \item two epistemic types:
    \begin{enumerate}
    \item probabilistic beliefs (assumed to be flat)
    \item non-probabilistic beliefs
    \end{enumerate}
  \item two preferences types (references to ``evolution of preference'' literature):
    \begin{enumerate}
    \item veridical payoff representations
    \item regret-based payoffs
    \end{enumerate}
  \item many other types are conceivable; more in discussion section
    \begin{itemize}
    \item we want to keep it simple
    \item assume that agents are relatively simple as well
    \item no learning from past encounters \& no reasoning about the opponent: agents simply do
      not know that there is another player (no strategic thinking possible; in line with idea
      of evolutionary game theory where agents are ``stupid'' and ``myopic'') \\
      \textcolor{blue}{this is then the added assumption that may be crude and generally
        implausible but allows us to deflect the economists' critique}
    \end{itemize}
  \item choice principles yield action choices for each game
    \begin{itemize}
    \item show how
    \end{itemize}
  \item RESULT: types (prob., veridical) and (prob., regret) are choice-equivalent (also
    mentioned by \citet{HalpernPass2012:Iterated-Regret})
  \item RESULT: in $2 \times 2$ cases, (prob., veridical), (prob., regret) and (non-prob.,
    regret) are choice-equivalent (but not for $n \times n$ in general)
    \begin{itemize}
    \item still we treat them as separate types because they behave differently under
      component-wise mutation (see below)
    \end{itemize}
  \end{itemize}
\end{itemize}

\subsection{The environment: meta-games}

\begin{itemize}
\item imagine huge, virtually infinite population of agents, each carrying a choice principle
\item agents play a randomly sampled game against a randomly sampled opponent
\item fitness is based on the average payoff of choice principle A against choice principle B,
  taken over arbitrary games $g$ and weighted their occurrence probability $P(g)$:
  $\int_g P(g) \times U_g(\text{CP-A}, \text{CP-B}) \text{d}g$
\item the occurrence probability of games $g$ is the ``environment'' and, possibly, impossible to
  derive theoretically, or to measure empirically
\item here, use a relatively neutral approach instead that is practicable:
  \begin{itemize}
  \item games are individuated by their payoff matrices (no framing effects or similar)
  \item each entry in a payoff matrix is an i.~i.~d.~random variable
    \begin{itemize}
    \item approach still incorporates a bias (e.g., against strategic form representations of
      sequential games)
    \end{itemize}
  \end{itemize}
\item example: average payoffs from numerical simulation of ($2 \times 2$ symmetric games)
\item example: average payoffs from numerical simulation of ($n \times n$ symmetric games, with
  $n$ uniformly random (up to some upper-bound))
\end{itemize}

\subsection{Results}

\textcolor{gray}{keep this clean and concise, formal details (proofs) in appendix}

\subsubsection{$2 \times 2$ symmetric games}

\textcolor{gray}{we isolate this case because it is most basic, and we can offer analytical
  proofs as well}

\begin{itemize}
\item given previous results, there are really only two kinds of types to look at:
  \begin{enumerate}
  \item (prob., veridical), (prob., regret) and (non-prob., regret) \\ 
    \textcolor{gray}{[all choice-equivalent in this case]}
  \item (non-prob., veridical)
  \end{enumerate}
\item RESULT: the former class ``dominates'' the latter (our main result, Proposition 1 from
  TARK paper)
\item this means that evolutionary dynamics would weed out (non-prob., regret); every
  population composition between the remaining three is neutrally stable; all interior
  trajectories (of the replicator dynamic) converge to these
\item further results (from TARK):
  \begin{itemize}
  \item component-wise mutation: abundance of regret types
    \begin{itemize}
    \item different evolutionary paths despite the same behavior (contra the ``behavioral
      gambit'' \citep{FawcettHamblin2013:Exposing-the-be})
    \end{itemize}
  \item exogenously given correlated epistemic types: regret types dominate
  \item exogenously given uncorrelated epistemic types: regret types dominate
  \end{itemize}
\todo[inline]{work out these additional results more carefully; present graphs if possible}
\end{itemize}

\subsubsection{$n \times n$ symmetric games}

\todo[inline]{work out these additional results more carefully; present graphs if possible}

\section{Discussion}

\begin{itemize}
\item what we achieved
\item what could be done next
\end{itemize}




\printbibliography[heading=bibintoc]


\newpage

\section*{Snippets}

\begin{itemize}
\item We present an approach to studying evolutionary selection of \emph{choice principles},
  i.e., ways of choosing consistently across several games. In other words, we would like to
  look at the problem of ``rational choice'' from an evolutionary point of view: what choice
  principles would be promoted, which demoted in evolutionary competition amongst each other?

\item This depends on the environment; it depends on what choice situations occur how
  frequently. Maybe some choice principle is better in some situations, while another one is
  better in others. 


\item We argue that this ``meta-evolutionary'' approach could eventually help explain attested
  deviations of human decision making from the classical ideal: choice behavior evolved to be
  an ``ecologically rational'', near-optimal adaptation to the environment
  \citep{Anderson1990:The-Adaptive-Ch,Anderson1991:Is-human-cognit,GigerenzerGoldstein1996:Reasoning-the-F,ChaterOaksford2000:The-Rational-An}. This
  is inline with recent aspirations to provide an evolutionary rationale for decision making
  processes
  \citep[e.g.][]{HammersteinStevens2012:Six-Reasons-for,FawcettHamblin2013:Exposing-the-be}. Here,
  we seek to contribute to the formal machinery, an extension of evolutionary game theory, that
  allows to capture statistical properties of the environment
  \citep[cf.][]{McNamara2013:Towards-a-Riche}.
\end{itemize}

\end{document}
