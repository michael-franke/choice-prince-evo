\documentclass{llncs}

\usepackage[style=authoryear-comp, % Citation marks as [Jef65]
            natbib=true,      % Natbib-style cite macros \citeauthor &c.
            hyperref=true,    % Cites in pdf are links to bib
                              % (hyperref conf.)
            maxnames=3,       % truncate name lists if more than 2
                              % names appear
            doi=false,        % no doi's
            url=false,        % no url's
            sortcites=false,  % do NOT sort cites in the style of the
                              % bibliography 
            %backref=true     % insert backrefs in reference section
            ]{biblatex}
\bibliography{MyRefGlobal}

\usepackage{amsfonts}
\usepackage{dsfont}

% \usepackage[colorinlistoftodos,color=lightgray,bordercolor=blue,textsize=footnotesize]{todonotes}

% \usepackage{setspace}
% \doublespacing
% \usepackage[left=3.75cm,right=3.75cm,top=3.75cm,bottom=3.75cm]{geometry}

\title{On the Evolution of Choice Principles}
\titlerunning{On the Evolution of Choice Principles}
\author{Michael Franke\inst{1} \and Paolo Galeazzi \inst{2}}
\institute{Department of Linguistics, University of T\"{u}bingen \and
  Institute for Logic, Language and Computation, Universiteit van Amsterdam}
\date{}

\begin{document}

\maketitle

\begin{abstract}
  We compare level-$k$ utility maximization and level-$k$ regret
  minimization in evolutionary competition on one-shot plays of
  arbitrary strategic games. RESULTS \dots
\end{abstract}



The widely accepted normative standard of individual decision making
under uncertainty is expected utility maximization. An agent is
rational only if his choices maximize expected utility. In games with
several players, agents who believe that their opponents are rational
might thus rule out certain of the opponents' choices. This kind of
reasoning can be iterated, giving rise to predictions of rational
choice under arbitrarily deep nestings of mutual beliefs in
rationality. % Depending on the detailed specification of what it means
% to have a belief in the opponents' rationality, and to respond
% rationally to such beliefs, standard game-theoretic solution concepts
% are characterized by this nested Theory of Mind reasoning, most
% notably the iterated deletion of strongly dominated strategies in
% strategic games \citep[e.g.][for
% overview]{Perea2012:Epistemic-Game-}.

Empirical data on human solitary or interactive decision-making is at
odds with this idealized picture. For one, human decision makers
systematically deviate from the predictions of expected utility theory
\citep[e.g.][]{TverskyKahnemann1974:Judgement-under,TverskyKahnemann1981:The-Framing-of-}. For
another, human capability for higher-order Theory of Mind (ToM)
reasoning appears to be rather limited \citep[e.g.][inter
alia]{HoCamerer1998:Iterated-Domina,KeyzarLin2003:Limits-on-Theor,VerbruggeMol2008:Learning-to-App,DegenFranke2013:Cost-Based-Prag}.

The usual reactions to this discrepancy between idealized norm and
empirical observation is to either rethink the prescriptive axioms of
rational choice or, alternatively, to devise a descriptive theory to
capture the empirical facts of human psychology of decision-making and
strategic reasoning
\citep[e.g.][]{Camerer2003:Behavioral-Game,GlimcherCamerer2009:Neuroeconomics:}. The
approach taken here is different from either. We refer to a general
decision making rule as a \emph{choice principle}. Formally, a choice
principle is a function that takes a game as input and returns a set
of acts as the decision maker's choice. We then adopt an evolutionary
point of view and ask: which choice principles are successful on
average in recurrent competition with alternative choice principles
(including more or less limited ToM reasoning)?

\citet{Weerdde-WeerdVerbrugge2013:How-much-does-i} address a part of
this question by looking at agent-based simulations of utility
maximizers with different ToM reasoning capabilities when playing
repeatedly selected zero-sum games against each other. The simulations
of \citeauthor{Weerdde-WeerdVerbrugge2013:How-much-does-i} suggest
that there is a benefit to deeper ToM reasoning only up to a fairly
limited depth of reasoning. 

Extending and generalizing this line of investigation, we use
numerical simulations to approximate the expected payoff of agents
with a fixed choice principle who play many \emph{arbitrary} 2-player
strategic games, not just a small hand-picked selection. This way we
try to asses the general evolutionary benefit of a choice principle
across an unbiased sample of games, not just for those games that are
of particular technical, philosophical or historical
importance. Additionally, we compare not only different ToM reasoning
capability of utility maximizers, but also variation in the underlying
method of choice. Concretely, we look at level-$k$ utility
maximization and level-$k$ iterated regret minimization (IRM)
\citep{HalpernPass2012:Iterated-Regret}.\footnote{Unlike
  \citeauthor{Weerdde-WeerdVerbrugge2013:How-much-does-i}, we look at
  level-$k$ reasoning models
  \citep[e.g.][]{Crawford2003:Lying-for-Strat,Crawford2007:Lets-Talk-It-Ove},
  not cognitive hierarchy models
  \citep[e.g.][]{CamererHo2004:A-Cognitive-Hie}. The former assume
  that agents of level $k+1$ believe that the opponent is of level
  $k$, while the latter assume a belief in a level-$l$ opponent with
  $l \le k$. Moreover, we assume all games to be one-shot encounters,
  while \citeauthor{Weerdde-WeerdVerbrugge2013:How-much-does-i} look
  at repeated encounters with the same opponent, and the concomitant
  possibility of strategic learning.}  The motivation for looking at
IRM is threefold. For one, regret minimization has been suggested as a
conceptually appealing alternative to utility maximization and a
potentially promising descriptive theory of individual decision-making
\citep{LoomesSugden1982:Regret-Theory:-}. For another, we are able to
show that non-iterated regret minimization outperforms the closely
related MaxiMin security strategy (see below) in evolutionary settings
like the one investigated here, making regret minimization the
evolutionarily best representative of a security strategy that we are
aware of.\footnote{More concretely, so far we have been able to prove
  this result for the class of 2-player symmetric games, and to
  demonstrate that it holds more generally using numerical simulations
  like the one reported here.} Finally, IRM usually requires only few
iteration steps, making it a serious challenger of the standard theory
in terms of limited ToM reasoning as well
\citep[c.f.][]{HalpernPass2012:Iterated-Regret}.

The choice principles we compare are defined as follows. L-0 utility
maximizers have an arbitrary probabilistic belief about the opponent's
behavior and are indifferent between any act that maximizes expected
utility under this belief. L-$k+1$ utility maximizers have an
arbitrary probabilistic belief over all acts that an L-$k$ opponent
would be indifferent about. Formally, if $U$ is a matrix of utilities
for the row player, regret minimization is equivalent to playing
MaxiMin on the derived matrix $U'$, where $U'_{ij} = U_{ij} -
\textrm{max}_{k} U_{kj}$ is the negative regret of choosing act $i$
when the opponent chooses $j$. L-0 regret minimizers are indifferent
between any act that minimizes regret in this sense. An L-$k+1$ regret
minimizer chooses like an L-0 regret minimizer in the reduced game in
which only acts remain that row and column players would choose that
are L-$k$ regret minimizers
\citep[see][]{HalpernPass2012:Iterated-Regret}.

We randomly generate arbitrary strategic games and record the payoff
earned by each choice principle when playing against any other,
including itself. In general, the details of generating arbitrary
games are important to this approach. Some choice principles might be
better in certain classes of games, but not others. For that reason,
we used several algorithms for creating games that are as unbiased and
neutral as possible. We choose neutrality in the selection of games,
because we have at present no better answer to the main empirical
question that is relevant here, namely which kinds of interactive
decision-making situations were most frequent during the critical
stages of cognitive development. For illustration, this abstract
focuses on the results of a simple generic algorithm for constructing
arbitrary asymmetric strategic games with 2 players. The procedure
starts by first picking a random number between 2 and 10 as the number
of acts for each player, say $n$ and $m$. Then, we determine each
entry in the $n \times m$ and $m \times n$ utility matrices of each
player as an integer between 0 and 10, all independently and uniformly
at random.

 

\begin{table}
  \centering
\begin{tabular}{rrrrrrrrr}
  \hline
 & EU-L0 & EU-L1 & EU-L2 & EU-L3 & EU-L4 & RM-L0 & RM-L1 & RM-L2 \\ 
  \hline
  EU-L0 & 6.754 & 6.764 & 6.729 & 6.889 & 6.790 & 6.747 & 6.767 & 6.767  \\ 
  EU-L1 & 7.155 & 6.696 & 6.666 & 6.609 & 6.949 & 7.004 & 7.072 & 7.071  \\ 
  EU-L2 & 6.692 & 7.412 & 6.959 & 6.850 & 6.973 & 6.457 & 6.554 & 6.556  \\ 
  EU-L3 & 6.707 & 7.002 & 7.610 & 7.132 & 7.165 & 6.474 & 6.524 & 6.525  \\ 
  EU-L4 & 6.827 & 7.075 & 7.327 & 7.716 & 7.417 & 6.649 & 6.699 & 6.701  \\ 
  RM-L0 & 6.713 & 6.754 & 6.711 & 6.850 & 6.741 & 6.709 & 6.728 & 6.729  \\ 
  RM-L1 & 6.877 & 6.851 & 6.803 & 6.941 & 6.874 & 7.085 & 7.107 & 7.107  \\ 
  RM-L2 & 6.879 & 6.853 & 6.803 & 6.943 & 6.875 & 7.086 & 7.117 & 7.117  \\ 
   \hline
\end{tabular}
\caption{Average utilities of level-$k$ utility maximizers and
  level-$k$ regret minimizers over 10.000 arbitrary games (see main
  text). In our sample, regret minimizers of level $k>2$ are
  behaviorally equivalent to RM-L$2$ and are therefore omitted in this table. } 
  \label{tab:MaxUtil}
\end{table}
 
Table~\ref{tab:MaxUtil} gives the accumulated payoff averaged over
10.000 games sampled in this way, for $0 \le k \le 4$. The table lists
the average payoff for the row principle when playing against the
column principle and can be regarded as a symmetric ``meta-game'' that
captures the evolutionary competition between choice principles. The
only evolutionarily stable strategy (ESS) of this meta-game is EU-L4,
but it is clear (from more extensive simulations) that this is just
because we chose an arbitrary cutoff for depth of
reasoning. Generally, it appears that pure populations of EU-L$k$
players can always be invaded by EU-L$k+1$ players. But,
interestingly, some pure populations of EU-L$k$ players can also be
invaded by EU-L$l$ players, with $l<k$ (see
Table~\ref{tab:MaxUtil}). On the other hand, if $k\ge2$, then RM-L$k$
is a neutrally stable strategy (NSS): regret minimizers of a different
level $k\ge2$ would not be driven to extinction, but would also not
take over the population (because they are behaviorally equivalent in
all games from our random sample).

The results from this example generalize to other game-sampling
routines, as long as they are general and encompassing enough.  While
there is usually no $k$ at which EU-L$k$ is an ESS, there is a small
$k$ (almost always $k\le 3$) for which IRM has reached its fixed point
in all sampled games and RM-L$k'$ is an NSS for all $k' \ge k$. This
is a noteworthy result: on randomly sampled games, it is always
beneficial for utility maximizers to outsmart the opponent by ideally
one level of ToM reasoning; too much ToM outsmarting is harmful. For
IRM, levels of ToM reasoning higher than a few steps are
evolutionarily pointless.

More fine-grained results obtain under the natural hypothesis that ToM
reasoning incurs a cost $c(k) \in \mathds{R}$, proportional to
reasoning depth $k$. In that case, the properties of the cost function
are crucial. If costs are initially small and grow not too
superlinearly in $k$, it still holds that usually no EU-L$k$ is an ESS
because it can be invaded by EU-L$k+1$. However, invasion by EU-L$l$
for $l < k$ is now more prominent than without reasoning costs, so
that mixed populations containing low-level EU-L$k$ players are often
the only dynamic attractors, e.g., under the replicator dynamic. If
costs grow more pronouncedly superlinearly in $k$, there can be $k$
for which EU-L$k$ is an ESS in the meta-game restricted to utility
maximizing choice principles, but this requires very particular cost
functions and is not very likely. The most frequent outcome of
evolutionary competition between EU-L$k$ players under superlinear
costs is a mixed population of fairly low level-$k$ players. But all
of this only holds when RM-L$k$ is excluded from consideration. When
we also look at IMR, adding reasoning costs of any type results in
RM-L$k$ being an ESS for some very low $k$. If ToM reasoning costs
grow pronouncedly superlinearly, this ESS is almost always
unique. This means that, at least under certain assumptions about the
costs of ToM reasoning, the iterated security strategy IRM
evolutionarily outperforms level-$k$ utility maximization.

In sum, our simulation data show that shallow depth of reasoning is
plausibly evolutionarily advantageous in long run competition with
``deep ToM'' strategies. Regret minimization often outperforms utility
maximization in this evolutionary competition, especially when
reasoning costs are added. Although already insightful, our results
are only partial, because many interesting questions have not yet been
addressed. We mention just three obvious extensions for future
work. Firstly, sequential games might yield different results, because
they would induce more ties in strategic form, and hence more reasons
to apply deeper ToM reasoning. Secondly, the payoff of utility
maximization heavily depends on the type of belief about the
opponent's choice. Here we used arbitrary beliefs, but certain
restrictions on belief formation, e.g., using flat beliefs throughout,
might give utility maximizers a better payoff on average. Finally, it
would be interesting to extend our comparison to more choice
principles, e.g., ones including learning from repeated interactions
with the same players
\citep[][]{Weerdde-WeerdVerbrugge2013:How-much-does-i}, or choice
principles building on prospect theory, for example.

\printbibliography[heading=bibintoc]

\end{document}
